%%%%%%%%%%%%%%%%%%%%%%%%%%%%%%%%%%%%%%%%%%%%%%%%%%%%%%%%%%%%%%%%%%%%%%%%%%%%%%%%%%%%%%%%%%%%%%%%%%%%%%%%%%%%%%%%%%%%%%%%%%%%%%%%%%%%%%%%%%%%%%%%%%%%%%%%%%%
% This is just an example/guide for you to refer to when submitting manuscripts to Frontiers, it is not mandatory to use Frontiers .cls files nor frontiers.tex  %
% This will only generate the Manuscript, the final article will be typeset by Frontiers after acceptance.   
% When submitting your files, remember to upload this *tex file, the pdf generated with it, the *bib file (if bibliography is not within the *tex) and all the figures.
%%%%%%%%%%%%%%%%%%%%%%%%%%%%%%%%%%%%%%%%%%%%%%%%%%%%%%%%%%%%%%%%%%%%%%%%%%%%%%%%%%%%%%%%%%%%%%%%%%%%%%%%%%%%%%%%%%%%%%%%%%%%%%%%%%%%%%%%%%%%%%%%%%%%%%%%%%%

%%% Version 3.4 Generated 2018/06/15 %%%
%%% You will need to have the following packages installed: datetime, fmtcount, etoolbox, fcprefix, which are normally inlcuded in WinEdt. %%%
%%% In http://www.ctan.org/ you can find the packages and how to install them, if necessary.
%%%  NB logo1.jpg is required in the path in order to correctly compile front page header %%%

\documentclass[utf8]{frontiersSCNS} % for Science, Engineering and Humanities and Social Sciences articles
%\documentclass[utf8]{frontiersHLTH} % for Health articles
%\documentclass[utf8]{frontiersFPHY} % for Physics and Applied Mathematics and Statistics articles

%\setcitestyle{square} % for Physics and Applied Mathematics and Statistics articles
\usepackage{url,hyperref,lineno,microtype,subcaption}
\usepackage[onehalfspacing]{setspace}
\usepackage{adjustbox}
\usepackage{graphicx}

\linenumbers

% Leave a blank line between paragraphs instead of using \\

\def\keyFont{\fontsize{8}{11}\helveticabold }
\def\firstAuthorLast{Sample {et~al.}} %use et al only if is more than 1 author
\def\Authors{First Author\,$^{1,*}$, Co-Author\,$^{2}$ and Co-Author\,$^{1,2}$}
% Affiliations should be keyed to the author's name with superscript numbers and be listed as follows: Laboratory, Institute, Department, Organization, City, State abbreviation (USA, Canada, Australia), and Country (without detailed address information such as city zip codes or street names).
% If one of the authors has a change of address, list the new address below the correspondence details using a superscript symbol and use the same symbol to indicate the author in the author list.
\def\Address{$^{1}$Laboratory X, Institute X, Department X, Organization X, City X , State XX (only USA, Canada and Australia), Country X \\
$^{2}$Laboratory X, Institute X, Department X, Organization X, City X , State XX (only USA, Canada and Australia), Country X  }
% The Corresponding Author should be marked with an asterisk
% Provide the exact contact address (this time including street name and city zip code) and email of the corresponding author
\def\corrAuthor{Corresponding Author}

\def\corrEmail{email@uni.edu}

\begin{document}
\onecolumn
\firstpage{1}

\title[Running Title]{A Hierarchical Modeling Analysis of Glucose Effectiveness in Subjects with Normal Glucose Tolerance and Type 2 Diabetes} 

\author[\firstAuthorLast ]{\Authors} %This field will be automatically populated
\address{} %This field will be automatically populated
\correspondance{} %This field will be automatically populated

\extraAuth{}% If there are more than 1 corresponding author, comment this line and uncomment the next one.
%\extraAuth{corresponding Author2 \\ Laboratory X2, Institute X2, Department X2, Organization X2, Street X2, City X2 , State XX2 (only USA, Canada and Australia), Zip Code2, X2 Country X2, email2@uni2.edu}

\maketitle
Number of words:  \\
Number of figures and tables: 9

\begin{abstract}
\section{}
abstract contents
% All article types require a minimum of 5 and a maximum of 8 keywords.

% For full guidelines regarding your manuscript please refer to \href{http://www.frontiersin.org/about/AuthorGuidelines}{Author Guidelines}.
% As a primary goal, the abstract should render the general significance and conceptual advance of the work clearly accessible to a broad readership. References should not be cited in the abstract. Leave the Abstract empty if your article does not require one, please see \href{http://www.frontiersin.org/about/AuthorGuidelines#SummaryTable}{Summary Table} for details according to article type. 

\tiny
 \keyFont{ \section{Keywords:} keyword, keyword, keyword, keyword, keyword, keyword, keyword, keyword} %All article types: you may provide up to 8 keywords; at least 5 are mandatory.
\end{abstract}

\section{INTRODUCTION} % Succinct, with no subheadings.
% glucose intolerance, SG importance, modulation of SG and its impact, physiologic of SG, therapy based on SG
Glucose intolerance, as determined by a combined interaction of insulin secretion, insulin action and glucose effectiveness ($S_G$), is a universal characteristic of both type 1 diabetes (T1D) and type 2 diabetes (T2D). Among the three components, glucose effectiveness is defined as the ability of glucose itself to increase glucose utilization and inhibit hepatic glucose production, via mass action effects and other mechanisms \citep{Dube2015}. Compared to insulin effects, glucose effectiveness exerts earlier influence to maintain normoglycemia. It was reported that, in normal subjects, glucose effectiveness accounts for 45\% to 65\% of the total glucose disposal after an intravenous glucose load \citep{Alford_2018}. In patients with defective insulin production and action (e.g., diabetes), the contribution of insulin is minimal and thus, $S_G$ mediated glucose disposition will further dominate \citep{Dube2015}. The potential application of $S_G$ as the predictor and therapeutic target of glucose intolerance and T2D has been mentioned before (\citet{basu_2009}, \citet{pau_2014}, \citet{Alford_2018}). However, previous research has been mainly focused on impaired insulin secretion function and insulin resistance as important factors of glucose intolerance or diabetes, rather than glucose effectiveness \citep{Alford_2018}. Although limited investigations that quantify $S_G$ have reported decreased $S_G$ in T2D, there still exists some inconsistencies, probably due to the differences in experiments and limited subjects included in the analysis \citep{Dube2015}. 

% measurements and population approach, previous papers
% short IVGTT paper, cite it
% working on it
Glucose clamp approach and minimal model (MM) approach have been commonly utilized to quantify glucose effectiveness. The glucose clamp approach involves controlling insulin at near-basal level so that the effects of glucose on its metabolism can be measured. Although this method is regarded as the gold standard of glucose kinetics evaluation, it requires cumbersome experiments and trained research teams. In contrast, minimal model approach is a modeling analysis that is based on intravenous glucose tolerance test (IVGTT). MM method provides a simpler and reliable solution to obtain glucose disposition indices, including glucose effectiveness ($S_G$) and insulin sensitivity ($S_I$). In addition, insulin-modified IVGTT (IM-IVGTT) has also been used to offer more insulin dynamics information than traditional IVGTT \citep{Vicini1999}. More recently, a short IVGTT lasting only 1 hr has been proposed to assess glucose effectiveness by the minimal model \citep{Morettini2018}. 

Hierarchical or population modeling, which is used widely in drug development, provides a formal basis for characterizing the distribution of model parameters in a population (central tendance and dispersion) and identifying relevant covariates that may explain aspects of the population parameter distribution (see \citet{Bonate2011}). Hierarchical modeling has been more recently employed for the glucose-insulin system and demonstrates improved estimation performance than non-population methods (\citet{Denti2009a}, \citet{Denti2010}). For example, the report by \citet{Denti2010} selected significant physiological features (covariates) to explain part of the inter-individual variability (IIV) in estimated minimal model parameters and integrated them into the final model to enhance the predictive power. It should be noted that, their analysis was conducted with the data of 204 healthy subjects after IM-IVGTT. Also, they used a simplified covariance matrix with correlation in only two pairs of parameters. % unclear, mixed results; any other previous papers? Mention if not

%  meta-analysis, pooled study, algorithm
In this study, we conducted a hierarchical modeling analysis of approximately 500 subjects with normal glucose tolerance and type 2 diabetes, using the glucose-insulin data pooled from more than 20 studies. Glucose effectiveness at zero insulin ($GEZI$) was estimated as one of the model parameters to evaluate its typical value and variation in subjects with different glucose tolerance conditions. Covariates such as weight, height, age, sex were assessed to explain the differences in $GEZI$ and other minimal model parameters. The covariate-parameter relationships were then used to simulate glucose profiles in virtual subjects with altered glucose effectiveness and insulin action. 
\section{MATERIALS AND METHODS}
\subsection{Clinical study data}
% In this study, we pooled the glucose-insulin concentration-time data of 497 subjects from 51 groups, with a total of 8024 measurements . In Table. \ref{tab:studies}, we summarized the pooled studies by each study group, with the cohorts, sex information, mean demographic characteristics, study types and references displayed. The whole dataset includes 228 healthy subjects, 115 obese subjects and 154 type 2 diabetes patients (T2D). Obese subjects were classified as people in good general health with a body mass index (BMI) larger than 28 $kg/m^2$. Patients in the T2D cohort had type 2 diabetes at the baseline or had a placebo administration. 229 of the subjects underwent an insulin-modified IVGTT and 268 had an IVGTT. For the subjects with missing continuous covariates, mean values reported in the original papers were used if available (Table. \ref{tab:studies}). For 40 subjects from study groups 31, 32, 33, 34, the values of height, weight and BMI were missing, and only the ranges of BMI were reported. We used a virtual population generator (PopGen, \citet{McNally2015}) to generate virtual subjects for these four groups and applied the calculated mean weight to those subjects. Then, the heights were imputed using the equation: $H=\sqrt{weight/BMI}$ and the given mean BMI value. The sex information of 18 subjects from two study groups 6 and 36 was missing but the sex ratio (i.e., male/female) was mentioned in the literature, so random assignment was used to fill in the sex value. For 41 subjects from groups 9, 10, 28, 29, 40, no sex information about gender was provided and their sex values were set as not available (NA). 
This study involves a pooled analysis of data from previous studies, each performed following the Declaration of Helsinki and upon approval of the respective institutional ethics committees, in which subjects were administered either an intravenous glucose tolerance test (IVGTT) or an insulin modified intravenous glucose tolerance test (IM-IVGTT). A total of 44 study groups was included in the analysis, comprising 497 different subjects as summarized in Table \ref{tab:studies}. Subjects with normal glucose tolerance (NGT) and with type 2 diabetes (T2D) (according to the guidelines in \citet{T2D_class}) were incorporated in the analysis, including both obese (body mass index (BMI) $>$ 28 $kg/m^2$) and non-obese subjects, but not subjects with other conditions that might alter glucose regulation. A standard IVGTT was performed in 229 subjects, while an IM-IVGTT was administered to 268 subjects. Table \ref{tab:studies} also summarizes the information on the sex, age, weight, height, and BMI of the subjects.

Studies in which some of these subject characteristics are missing in individual subjects are noted in Table \ref{tab:studies}, with missing values imputed as follows. For studies in which only mean values of age, weight, height or BMI were reported, each subject was assigned the corresponding mean value from that study as noted in Table \ref{tab:studies}. For 40 subjects from four of the study groups, the values of height, weight and BMI were missing, and only the mean and standard deviation (SD) of BMI were reported. For these subjects, we applied a virtual population generator PopGen (\cite{McNally2015}) to produce 40 virtual subjects. For the inputs, we used mean BMI±2SD as the BMI range, reported mean age, 100 cm to 200 cm as the height range, reported proportion of males and set all the subjects as whites based on the original paper \citep{1998_AGING_Ahren}. The mean body weight of virtual subjects in each group was applied. Then heights were calculated using the equation: $H=\sqrt{weight/BMI}$ and the mean BMI value. The sex of 18 subjects from two study groups was missing but the proportion of men and woman was reported, which was used to randomly assign the sex of the subjects. For 41 subjects from five studies, no sex was provided, and the sex of these subjects was classified as not available (NA). \\

\subsection{Minimal model}
% We employed M1 version of the minimal model (\citet{Bergman1979}, \citet{Araujo-vilar1998}) to describe glucose-insulin disposition processes. The model equations can be showed as:
The following parameterization of the minimal model glucose and insulin was used in the analysis (\citet{Bergman1979}, \citet{Araujo-vilar1998}):
\begin{equation}
\frac{dG(t)}{dt} =-(GEZI+X(t))*G(t)+(GEZI+X_{basal})*G_{basal} \quad G_{(t=0)} =G_{basal}+\frac{Dose}{V}\label{eq:01}
\end{equation}
\begin{equation}
\frac{dX(t)}{dt} =-p2*X(t)+p2*S_I*I(t) \quad X_{basal} =S_I*I_{basal}\label{eq:02}
\end{equation}
where $Dose$ denotes the glucose dose (mmol) at time zero, $G(t)$ is the plasma glucose concentration (mmol/L), $G_{basal}$ is basal glucose concentration (measured glucose at end point), $X(t)$ is the remote insulin action (min$^{-1}$), $I(t)$ is the measured plasma insulin concentration (pmol/L) and $I_{basal}$ is the basal insulin concentration (measured insulin at end point). $I(t)$ was set as known and error-free inputs of the model. Model parameters are: glucose effectiveness at zero insulin ($GEZI$, min$^{-1}$), insulin sensitivity ($S_I$, min$^{-1}$/(pmol/L)), remote insulin action parameter ($p2$, min$^{-1}$) and the volume of glucose distribution ($V$, L). For IVGTT, a glucose dose of 0.3 g/kg was administrated to the subjects at time zero. The study duration ranges between 180 min and 360 min and the number of samples ranges from 12 to 30 for each subject. For IM-IVGTT, the same glucose dose was given for each subject and a short insulin infusion of 0.03 to 0.05 U/kg at 20 min was given, with the duration within the range of 180 min and 240 min and number of samples ranging from 12 to 22. The glucose measurements prior to 5 min were excluded from the analysis, since the one-compartment glucose kinetic model does not represent the initial phase of glucose disposition  \citep{Vicini1999}.

\subsection{Hierarchical modeling analysis}
% The minimal model (M1 version) was used as our base model and physiological covariates were evaluated to account for inter-individual variability (IIV) in model parameter estimates. We first analyzed the glucose concentration data of healthy, obese and T2D cohorts separately, using the maximum-likelihood, expectation maximization (MLEM) program in ADAPT software (Version 5) \citep{AdaptUserGuide}. The four model parameters were assumed to follow a multivariate log-normal distribution around a typical value, with a full covariance matrix. We used a proportional error model to describe the residual errors. Individual random effects of parameter estimates were obtained and plotted versus physiological covariates to explore their relations preliminarily. 
% Next, we combined the data from the three cohorts and the base model was fitted to the data. The development of covariate models was based on preliminary relations in different cohorts, previous reports, scientific interest and mechanistic plausibility. The covariates were selected based on good estimate precision and improved objection function value (-2 log likelihood) via forward addition and backward elimination (p$<$0.01). We tested potential covariate models for $S_I$ first, followed by $V$, $GEZI$ and $p2$. For continuous covariates (i.e., weight, height, BMI, age), power models centered at their median values were used. For discrete covariates (i.e., sex), proportional changes in the typical value and power term were investigated in the modeling process. The random effects of model parameters for the final model were checked against covariates to evaluate the remaining individual variation around the typical values.
Hierarchical modeling, which can better quantify both individual parameter estimates and inter-individual variability (IIV) in parameters \citep{Bonate2011}. One notable application of population modeling to the glucose-insulin system in NGT subjects was reported by \citet{Denti2010}. 

In this work, Eqs (\ref{eq:01}) and (\ref{eq:02}) define the first stage of the hierarchical framework, where the residual error (defined as the difference between the measured and predicted glucose concentrations) was assumed to be normally distributed with variance proportional to the predicted glucose concentration. For the second stage of the hierarchy, the vector of model parameters, $log[GEZI, SI, p2, V]$ is assumed to follow a multivariate normal distribution with mean and covariance to be estimated from the pooled study data. Maximum likelihood estimates of the model population parameters were obtained using the expectation maximization (MLEM) algorithm in ADAPT (Version 5) software \citep{AdaptUserGuide}. 

The following covariates were examined for their influence on the model parameters: age, body weight, height, BMI, sex, and glucose tolerance (NGT/T2D). Covariate-parameter relationships were identified based on exploratory graphical analysis and mechanistic plausibility. The covariates were selected based on model estimate precision and improved objection function value (-2 log likelihood) via forward addition and backward elimination (p$<$0.01). We tested covariate models for $S_I$ initially, followed by $V$, $GEZI$ and $p2$. For continuous covariates (age, body weight, height, BMI), power models centered at their median values of the covariates were used. For categorical covariates (sex and glucose tolerance), changes in the covariate models parameters between categories were investigated. 

\section{RESULTS}
% covariate relationships (fig. 1), table of prm estimates, diagnostic plots (fittings, comparison, residual error), model equation
After missing covariate imputation, demographic characteristics of 497 subjects were summarized in Table. \ref{tab:demo}, which includes sex, age, weight, height and body mass index. A graphic overview of investigated covariates is provided in Figure \ref{fig: cova}. 

Using the data of each of the three cohorts, weight-related indexes (body weight, BMI) were found to negatively associate with $S_I$ (results not shown), which is consistent with the previous investigation \citep{Bergman1997TheTolerance}. Also, weight-related indexes (height, weight, BMI) are positively correlated with V (results not shown). This is also in line with a previous report \citep{Denti2010}.

After pooling all the data together, the following minimal model with covariates was selected to describe the insulin-glucose profiles: $GEZI=0.0211*(1-0.480*T2D)$; for NGT subjects, $T2D$ is 0; for T2D subjects, $T2D$ is 1; $S_I=(5.59e-05)*(1-0.575*T2D)*(BMI/25.335)^{-2.12}$; BMI is in $kg/m^2$; $p2=0.0428$; $V=12.0*(weight/75)^{0.851}$; weight is in kg. The incorporation of these covariates into the model reduces the inter-individual variability of $GEZI$, $S_I$ and $V$ from 50.9\%,  113\% and 34.4\% to 46.5\%, 88.6\% and 26.8\%, respectively (Table. \ref{tab:prm estimates}). Model parameter were well estimated, with relative standard error less than 9\%. Quantile-quantile plots and histograms suggest the four model parameters basically follow a log-normal distribution (figures not shown). The goodness-of-fit plots (Figure \ref{fig: fittings}) imply that the final model can fully describe the observed glucose concentrations, without significant biases. The upper row of Figure \ref{fig: fittings} compares the population prediction of the base model (Figure \ref{fig: fittings}(A)) and our final covariate model (Figure \ref{fig: fittings}(B)), suggesting an improved description of the observed data.

\subsection{$GEZI$ is decreased in T2D, independent of BMI}
The distribution of $GEZI$ in the NGT and T2D subjects are shown in Figure \ref{fig: SG_co}. %Similar values of $GEZI$ in control and obese subjects can be observed, which indicates extra adipose has no significant influence on glucose disposition mediated by itself. 
T2D cohort displays a significantly lower $GEZI$ value (48\% lower) compared to the NGT cohort. A likelihood-ratio test (LRT) was conducted to assess the necessity of considering cohort difference in $GEZI$ typical values, with the null hypothesis as "no cohort difference in $GEZI$ in the model". LRT is 100.5 between the final model and the competing model, with a p value of 1.18e-23, suggesting a distinct effect of cohort on $GEZI$. No other covariate was found to affect $GEZI$, which is aligned with a previous report \citep{Denti2010}.

\subsection{$S_I$ is decreased in T2D and depends on BMI}
The scatter plots of estimated $S_I$ and BMI of all the subjects are shown in Figure \ref{fig: SI_BMI}, with model predicted curves of two sub-populations (NGT, T2D) superimposed. In both of the sub-populations, higher BMI values lead to decreased $S_I$ (with a power of -2.12). This is consistent with the conclusions in \citet{Bergman1997TheTolerance}, in which they reported a negative association between BMI and $S_I$ and reduced $S_I$ values in obese subjects. Besides, type 2 diabetes subjects have significantly decreased (about 58\% lower) $S_I$ given the same BMI. But the relationship between $S_I$ and BMI is not altered by T2D. This confirms that $S_I$ can serve as an indicator or therapeutic target of type 2 diabetes or other related diseases. Besides the covariate models for $GEZI$ and $S_I$, the volume of glucose distribution (with a typical value of 12 L) has a positive correlation with the body weight of subjects (power of 0.85), without any difference in the two cohorts (Figure \ref{fig: V_BW}). This positive association is aligned with the previous report \citep{Denti2010}, in which they normalized $V$ using subjects' body weights. 

\subsection{Impaired $GEZI$ in T2D has stronger influence on glucose concentration than $S_I$}
In order to compare the relative contribution of $GEZI$ and $S_I$ on controlling glucose concentration, we simulated the glucose profiles of four virtual subjects with different $GEZI$ and $S_I$ values during an IVGTT (Figure \ref{fig: simu}). We fixed the body weight and BMI to the median value of all subjects, so that the virtual subjects had the same $p2$ and volume of distribution ($V$) without any effects by BMI or body weight. Compared to the healthy subject A, $GEZI$ and $S_I$ were reduced due to the development of T2D in other virtual subjects. The most impaired function of controlling glucose can be observed in the virtual subject B (a typical T2D patient) (Figure \ref{fig: simu}). The comparison of glucose profiles of subject C and D indicates the decrease of $GEZI$ due to T2D results in stronger influence than the corresponding reduction in $S_I$, which suggests the important role of glucose effectiveness in the regulation of glucose homeostasis \citep{Dube2015}. 

\section{DISCUSSION}
% the increase of SG in normal "at risk" subjects, but reduced SG in estbalished subjects (Alford_2018); Over time, the combination of a low GE and disordered β‐cell function leads to glucose intolerance and type 2 diabetes.
% mention remaining random effects of SG, SI, p2, V
% mention variability of SI
% SGLT2 inhibitor on SG: Morettini2018
% SG is a hybrid parameter, describing at the same time the effect of glucose per se on glucose disposal and production and the exchange kinetics between the two glucose compartments. Thus, care should be exercised when interpreting its physiological signif- icance: from Vicini1999
% discuss and compare with the results in Cobelli paper


% This section may be divided by subheadings. Discussions should cover the key findings of the study: discuss any prior research related to the subject to place the novelty of the discovery in the appropriate context, discuss the potential shortcomings and limitations on their interpretations, discuss their integration into the current understanding of the problem and how this advances the current views, speculate on the future direction of the research, and freely postulate theories that could be tested in the future.

% \section*{Article types}

%  Original Research articles are peer-reviewed, have a maximum word count of 12,000 and may contain no more than 15 Figures/Tables. Authors are required to pay a fee (A-type article) to publish an Original Research article. Original Research articles should have the following format: 1) Abstract, 2) Introduction, 3) Materials and Methods, 4) Results, 5) Discussion.

% For requirements for a specific article type please refer to the Article Types on any Frontiers journal page. Please also refer to  \href{http://home.frontiersin.org/about/author-guidelines#Sections}{Author Guidelines} for further information on how to organize your manuscript in the required sections or their equivalents for your field

% For Original Research articles, please note that the Material and Methods section can be placed in any of the following ways: before Results, before Discussion or after Discussion.

% \section{Manuscript Formatting}

% \subsection{Heading Levels}

%There are 5 heading levels

%\subsection{Level 2}
%\subsubsection{Level 3}
%\paragraph{Level 4}
%\subparagraph{Level 5}

%\subsection{Equations}
% Equations should be inserted in editable format from the equation editor.

% \subsection{Figures}
% Frontiers requires figures to be submitted individually, in the same order as they are referred to in the manuscript. Figures will then be automatically embedded at the bottom of the submitted manuscript. Kindly ensure that each table and figure is mentioned in the text and in numerical order. Figures must be of sufficient resolution for publication \href{http://home.frontiersin.org/about/author-guidelines#ResolutionRequirements}{see here for examples and minimum requirements}. Figures which are not according to the guidelines will cause substantial delay during the production process. Please see \href{http://home.frontiersin.org/about/author-guidelines#GeneralStyleGuidelinesforFigures}{here} for full figure guidelines. Cite figures with subfigures as figure \ref{fig:2}B.

% \subsubsection{Permission to Reuse and Copyright}
% Figures, tables, and images will be published under a Creative Commons CC-BY licence and permission must be obtained for use of copyrighted material from other sources (including re-published/adapted/modified/partial figures and images from the internet). It is the responsibility of the authors to acquire the licenses, to follow any citation instructions requested by third-party rights holders, and cover any supplementary charges.
%%Figures, tables, and images will be published under a Creative Commons CC-BY licence and permission must be obtained for use of copyrighted material from other sources (including re-published/adapted/modified/partial figures and images from the internet). It is the responsibility of the authors to acquire the licenses, to follow any citation instructions requested by third-party rights holders, and cover any supplementary charges.

% \subsection{Tables}
% Please note that very large tables (covering several pages) cannot be included in the final PDF for reasons of space. These tables will be published as \href{http://home.frontiersin.org/about/author-guidelines#SupplementaryMaterial}{Supplementary Material} on the online article page at the time of acceptance. The author will be notified during the typesetting of the final article if this is the case. 

\section*{Nomenclature}

\subsection*{Resource Identification Initiative}
% To take part in the Resource Identification Initiative, please use the corresponding catalog number and RRID in your current manuscript. For more information about the project and for steps on how to search for an RRID, please click \href{http://www.frontiersin.org/files/pdf/letter_to_author.pdf}{here}.

\subsection*{Life Science Identifiers}
% Life Science Identifiers (LSIDs) for ZOOBANK registered names or nomenclatural acts should be listed in the manuscript before the keywords. For more information on LSIDs please see \href{http://www.frontiersin.org/about/AuthorGuidelines#InclusionofZoologicalNomenclature}{Inclusion of Zoological Nomenclature} section of the guidelines.

\section*{Additional Requirements}
% For additional requirements for specific article types and further information please refer to \href{http://www.frontiersin.org/about/AuthorGuidelines#AdditionalRequirements}{Author Guidelines}.

\section*{Conflict of Interest Statement}
%All financial, commercial or other relationships that might be perceived by the academic community as representing a potential conflict of interest must be disclosed. If no such relationship exists, authors will be asked to confirm the following statement: 
% The authors declare that the research was conducted in the absence of any commercial or financial relationships that could be construed as a potential conflict of interest.

\section*{Author Contributions}
% The Author Contributions section is mandatory for all articles, including articles by sole authors. If an appropriate statement is not provided on submission, a standard one will be inserted during the production process. The Author Contributions statement must describe the contributions of individual authors referred to by their initials and, in doing so, all authors agree to be accountable for the content of the work. Please see  \href{http://home.frontiersin.org/about/author-guidelines#AuthorandContributors}{here} for full authorship criteria.

\section*{Funding}
% Details of all funding sources should be provided, including grant numbers if applicable. Please ensure to add all necessary funding information, as after publication this is no longer possible.

\section*{Acknowledgments}
% This is a short text to acknowledge the contributions of specific colleagues, institutions, or agencies that aided the efforts of the authors.

\section*{Supplemental Data}
 % \href{http://home.frontiersin.org/about/author-guidelines#SupplementaryMaterial}{Supplementary Material} should be uploaded separately on submission, if there are Supplementary Figures, please include the caption in the same file as the figure. LaTeX Supplementary Material templates can be found in the Frontiers LaTeX folder.

\section*{Data Availability Statement}
The datasets [GENERATED/ANALYZED] for this study can be found in the [NAME OF REPOhttps://www.overleaf.com/project/5f7c9af1716f7400018a1fe7SITORY] [LINK].
% Please see the availability of data guidelines for more information, at https://www.frontiersin.org/about/author-guidelines#AvailabilityofData

\bibliographystyle{frontiersinSCNS_ENG_HUMS} % for Science, Engineering and Humanities and Social Sciences articles, for Humanities and Social Sciences articles please include page numbers in the in-text citations
%\bibliographystyle{frontiersinHLTH&FPHY} % for Health, Physics and Mathematics articles
\bibliography{references}

%%% Make sure to upload the bib file along with the tex file and PDF
%%% Please see the test.bib file for some examples of references

\begin{table}[h]
\caption{Summary of subject characteristics in the studies (mean±SD)}
\label{tab:studies}
\scalebox{0.7}{
\begin{tabular}{llllllllll}
\hline
Study No. & No. of subjects & Cohort & Sex (F/M/NA) & Age (yrs) & Weight (kg) & BMI (kg/m^2) & Height (cm) & Study type & Reference \\ \hline
1  & 9  & T2D & 0/9/0   & 62.1±5.16   & 73.1±11.1 & 28.3±4.48   & 161±7.95   & IM-IVGTT & \citet{2001_Gemfibrozil_Avogaro} \\
2  & 9  & NGT & 3/6/0   & 27.6±9.44   & 68.3±10.9 & 22.3±3.39   & 175±7.18   & IM-IVGTT & \citet{2002_VODKA_Avogaro-cnt}  \\
3  & 8  & NGT & 1/7/0   & 52.5±2.98   & 85.8±18.1 & 28.9±6.7    & 173±4.44   & IM-IVGTT & \citet{2004_Vodka_Avogaro-T2}    \\
4  & 8  & T2D & 1/7/0 $^a$ & 64.5±6.26   & 88.4±10.6 & 29.3±2.54   & 173±6.16   & IM-IVGTT & \citet{2004_Vodka_Avogaro-T2}    \\
5  & 6  & T2D & 0/6/0   & 57.0±7.92     & 92.1±8.45 & 29.2±1.9    & 178±5.05   & IM-IVGTT & \citet{2003_CAIAPO_Ludvik}       \\
6  & 18 & T2D & 0/18/0  & 57.7±8.11   & 88.3±12   & 27.8±2.72   & 178±6.65   & IM-IVGTT & \citet{2003_CAIAPO_Ludvik}       \\
7  & 11 & NGT & 1/1/11  & 29.0±0 $^b$      & 67.7±5.88 & 22.5±0 $^b$    & 173±7.56 $^d$ & IVGTT    & \citet{1999_alcohol_avogaro}     \\
8  & 31 & T2D & 10/17/6 & 50.8±12.9   & 85.8±19.9 & 29.5±6.9    & 171±9.6    & IM-IVGTT & \citet{2008_GAD_Nolan}           \\
9  & 10 & T2D & 7/3/0   & 50.4±7.24   & 78.8±20.4 & 30.0±6.49     & 162±7.44   & IM-IVGTT & Not published  \\
10 & 2  & NGT & 2/0/0   & 29.0±9.9      & 100±17.3  & 35.2±8.67   & 170±6.36   & IM-IVGTT & Not published            \\
11 & 2  & T2D & 2/0/0   & 36.0±4.24     & 107±15.3  & 34.0±4.04     & 178±2.12   & IM-IVGTT & Not published            \\
12 & 10 & T2D & 4/6/0   & 66.0±4.71     & 64.3±7.45 & 23.8±0 $^b$    & 164±9.45 $^d$ & IVGTT    & \citet{Viviani1999}  \\
13 & 6  & NGT & 2/4/0   & 73.2±7.33   & 63.0±9.25   & 23.1±0 $^b$    & 165±12.2 $^d$ & IVGTT    & \citet{Viviani1999}  \\
14 & 11 & NGT & 1/10/0  & 24.6±7.21   & 71.5±13.7 & 23.7±0 $^b$    & 173±17.7 $^d$ & IVGTT    & \citet{Viviani1999}  \\
15 & 23 & T2D & 6/17/0  & 28.4±7.84   & 107±20.3  & 34.8±5.45   & 175±11.3   & IM-IVGTT & \citet{2005_YoungT2_Nolan}       \\
16 & 9  & NGT & 5/4/0   & 35.2±8.63   & 66.7±5.24 & 23.0±1.58     & 170±5.57   & IM-IVGTT & \citet{2005_YoungT2_Nolan}       \\
17 & 10 & NGT & 7/3/0   & 18.6±3.81   & 109±14.5  & 35.8±3.55   & 174±5.36   & IM-IVGTT & \citet{2005_YoungT2_Nolan}       \\
18 & 5  & T2D & 5/0/0   & 12.2±1.86   & 64.8±8.17 & 27.1±2.94   & 155±2.79   & IM-IVGTT & \citet{2005_YoungT2_Nolan}      \\
19 & 2  & NGT & 1/1/0   & 27.0±12.7     & 69.5±7.78 & 25.6±5.68   & 166±9.19   & IVGTT   & Not published  \\
20 & 15 & NGT & 7/8/0   & 38.9±10.8   & 68.8±12.3 & 24.3±2.6    & 168±10.6   & IM-IVGTT & \citet{1998_InsSens_Pacini}      \\
21 & 10 & NGT & 10/0/0  & 26.3±2.58   & 57.0±5.31   & 20.7±2.3    & 166±6.51   & IM-IVGTT & \citet{2005_PCOS_Gennarelli}     \\
22 & 10 & T2D & 4/6/0   & 57.8±8      & 69.0±9.98   & 25.3±1.8    & 165±8.95   & IVGTT     & Not published  \\
23 & 10 & T2D & 4/6/0   & 54.6±11.2   & 68.9±9.72 & 25.3±1.64   & 165±8.95   & IVGTT    & Not published  \\
24 & 13 & NGT & 1/1/13  & 68.3±5.42   & 71.7±8.73 & 24.6±1.96   & 171±5.33   & IVGTT     & \citet{pacini_1998}                      \\
25 & 10 & NGT & 1/1/10  & 26.7±2      & 72.3±9.71 & 22.9±2.89   & 178±5.87   & IVGTT    & \citet{pacini_1998}                      \\
26 & 10 & NGT & 2/8/0   & 36.1±9.61   & 71.2±7.1  & 23.8±2.03   & 173±3.35   & IVGTT    & \citet{nardi_1994}                      \\
27 & 10 & NGT & 10/0/0  & 27.0±0 $^b$      & 62.1±0 $^c$  & 24.9±0 $^c$    & 158±0 $^d$    & IVGTT    & \citet{1998_AGING_Ahren}         \\
28 & 10 & NGT & 10/0/0  & 63.0±0 $^b$      & 68.0±0 $^c$    & 25.2±0 $^c$    & 164±0 $^d$    & IVGTT    & \citet{1998_AGING_Ahren}         \\
29 & 10 & NGT & 0/10/0  & 27.0±0 $^b$      & 74.4±0 $^c$  & 24.9±0 $^c$    & 173±0 $^d$    & IVGTT    & \citet{1998_AGING_Ahren}        \\
30 & 10 & NGT & 0/10/0  & 63.0±0 $^b$      & 78.6±0 $^c$  & 25.2±0 $^c$    & 177±0 $^d$    & IVGTT    & \citet{1998_AGING_Ahren}         \\
31 & 9  & NGT & 7/2/0   & 17.0±2.24     & 54.2±9.08 & 19.7±2.5    & 165±8.37   & IVGTT   & \citet{pagano_1995}                      \\
32 & 10 & NGT & 2/8/0 $^a$ & 35.6±4.7    & 75.3±14.3 & 24.5±3.18   & 175±8.49   & IVGTT    & \citet{2001_MyocInfarct_Cavallo} \\
33 & 13 & NGT & 10/3/0  & 13.3±0.63   & 84.2±10.2 & 32.5±3.08   & 161±6.57   & IVGTT    & \citet{Cerutti1998}                      \\
34 & 4  & NGT & 1/3/0   & 32.2±11.2   & 75.8±10.7 & 23.9±1.06   & 178±9.54   & IM-IVGTT & \citet{2002_AGENESIS_Roden}      \\
35 & 9  & NGT & 6/4/1   & 43.9±0 $^b$    & 65.7±0 $^b$  & 24.1±0 $^b$    & 165±0 $^d$    & IVGTT & \citet{2008_hanisurya_thyroid} \\
36 & 38 & NGT & 38/0/0  & 31.5±5.55   & 68.4±13.3 & 25.0±5.68     & 166±5.15   & IM-IVGTT & \citet{2012_pGDM_tura}           \\
37 & 18 & NGT & 9/9/0   & 44.9±12.8 $^b$ & 114±23.3  & 39.4±3.57 $^b$ & 169±12.6 $^d$ & IVGTT    & \cite{Kautzky-Willer1992} \\
38 & 17 & NGT & 10/7/0  & 33.5±14.3   & 67.5±13.1 & 23.0±5.1      & 172±11.6   & IVGTT    & \citet{Kautzky-Willer1992}     \\
39 & 7  & NGT & 2/5/0   & 30.3±6.52   & 70.0±8.91   & 23.5±0.835  & 172±9.56   & IVGTT    & \citet{alex_1996}                     \\
40 & 12 & T2D & 0/12/0  & 64.0±5.88 $^b$   & 95±19.6   & 28.6±5.63 $^b$ & 182±8.38 $^d$ & IM-IVGTT & \citet{2010_alcohol_ludvik}      \\
41 & 17 & NGT & 17/0/0  & 38.1±7.85   & 84.3±11.7 & 33.4±4.05   & 159±6.02   & IVGTT     & \citet{2006_Davi_JACC}           \\
42 & 13 & NGT & 13/0/0  & 42.7±11.3   & 94.1±12.4 & 37.4±3.59   & 159±9.86   & IVGTT    & \citet{2006_Davi_JACC}           \\
43 & 11 & NGT & 11/0/0  & 45.9±7.61   & 111±15.9  & 44.7±5.82   & 158±2.66   & IVGTT    & \citet{2003_INFLAMMATORY_Davi}   \\
44 & 11 & NGT & 11/0/0  & 48.2±7.92   & 95.8±9.46 & 38.1±3.03   & 159±3.88   & IVGTT    & \citet{2003_INFLAMMATORY_Davi}  \\ \hline
\end{tabular}}
\\
The values in cells without superscript are known \\ 
$^a$: random assignment \\
$^b$: mean value \\
$^c$: PopGen \\
$^d$: calculated \\
\end{table}

% to do: replace reference

\begin{table}[h]
\caption{Demographic characteristics of the study subjects}
\label{tab:demo}
\begin{tabular}{llllll}
\hline
Characteristic                & No.         & Mean±SD     & Minimum & Median & Maximum \\ \hline
Cohort (NGT/T2D)   & 343/154 &             &         &        &         \\
Sex (female/male/missing)     & 239/217/41  &             &         &        &         \\
Age (yrs)                     &             & 41.4±16.9 & 9.70    & 40.0  & 86.0  \\
Weight (kg)                   &             & 79.7±19.9 & 40.0   & 75.0  & 157 \\
Height (cm)                   &             & 169±10.1 & 130  & 168 & 196 \\
BMI (kg/$m^2$) &             & 28.0±6.76  & 15.9  & 25.3  & 53.9  \\ \hline
\end{tabular}
\end{table}

\begin{table}[h]
\caption{Model parameter estimates}
\label{tab:prm estimates}
\begin{tabular}{lll}
\hline
Parameter (Unit)    & Estimate without covariates (relative SE) & Estimate with covariates (relative SE) \\ \hline
-2 log likelihood             & 18674.1       & 18137.8       \\
                              &               &               \\
$GEZI$ (min-1)                  & 0.0178 (3.37) & 0.0211 (3.68) \\
$S_I$ (min-1/(pmol/L)) & 3.59e-05 (5.80)                          & 5.59e-05 (6.02)                        \\
$p2$ (min-1)                    & 0.0425 (3.62) & 0.0428 (3.52) \\
$V$ (L)                         & 12.4 (1.87)   & 12.0 (1.56)   \\
                              &               &               \\
Proportional error  & 0.0706 (0.352)                            & 0.0705 (0.347)                         \\
                              &               &               \\
IIV in $GEZI$                      & 50.9 (4.65)   & 46.5 (4.99)   \\
IIV in $S_I$                        & 113 (3.83)    & 88.6 (2.91)   \\
IIV in $p2$                        & 44.0 (7.79)   & 44.4 (7.38)   \\
IIV in $V$                         & 34.4 (3.48)   & 26.8 (3.02)   \\
                              &               &               \\
Value representing effect of: &               &               \\
T2D on $GEZI$                  &               & -0.480 (8.42) \\
T2D on $S_I$                     &               & -0.575 (6.44) \\
BMI on $S_I$                     &               & -2.12 (8.86)  \\
BW on $V$                       &               & 0.851 (6.61)  \\ \hline
\end{tabular}
IIV: inter-individual variability\\
Correlation between model parameters: $GEZI$ and $S_I$: -0.20; $GEZI$ and $p2$: 0.77; $GEZI$ and $p2$: -0.07; $S_I$ and $p2$: -0.20; $S_I$ and $V$: 0.19; $p2$ and $V$: -0.32
\end{table}

%%% Please be aware that for original research articles we only permit a combined number of 15 figures and tables, one figure with multiple subfigures will count as only one figure.
%%% If using *.tif files convert them to .jpg or .png
%%%  NB logo1.eps is required in the path in order to correctly compile front page header %%%
%%% Frontiers will add the figures at the end of the provisional pdf automatically
\begin{figure}[h!]
\begin{center}
\includegraphics[width=15cm]{p2.PNG}
\end{center}
\caption{Overview of covariate values and relationships. Histograms plots for continuous covariates and bar graphs for discrete covariates are shown on the diagnoal line. In the lower triangle, the boxplots between continuous and discrete covariates and scatter plots between continuous covariates are displayed. In the upper triangle, the correlation coefficients between continuous covariates are shown}
\label{fig: cova}
\end{figure}

\begin{figure}[h!]
\begin{center}
\includegraphics[width=14cm]{comb.PNG}
\end{center}
\caption{Goodness-of-fit plots of the based and final model. \textbf{(A)}: observed glucose concentration versus population prediction via the base model.  \textbf{(B)}: observed glucose concentration versus population prediction via the final model. \textbf{(C)}: observed glucose concentration versus individual prediction via the final model. \textbf{(D)}: standardized residual versus population prediction via the final model. \textbf{(E)}: standardized residual of the final model versus time}
\label{fig: fittings}
\end{figure}

\begin{figure}[h!]
\begin{center}
\includegraphics[width=15cm]{SG_co.PNG}
\end{center}
\caption{Violin plots showing the distribution of $GEZI$ values in NGT and T2D cohorts. Boxplots were inserted for each cohort to indicate medians and interquartile ranges}
\label{fig: SG_co}
\end{figure}

\begin{figure}[h!]
\begin{center}
\includegraphics[width=15cm]{SI_BMI.PNG}
\end{center}
\caption{Scatter plots and model predicted relationships between $S_I$ and BMI in NGT and T2D cohorts. Dots with different colors display the negative association between $S_I$ and BMI. The black curve shows the model predicted $S_I$ given the BMI value in NGT subjects, with standard error of the prediction labelled. The red curve is the corresponding curve in T2D patients}
\label{fig: SI_BMI}
\end{figure}

\begin{figure}[h!]
\begin{center}
\includegraphics[width=15cm]{V_BW.PNG}
\end{center}
\caption{Scatter plots and model predicted relationships between $V$ and body weight in NGT and T2D cohorts. Dots with different colors display the positive association between $V$ and body weight. The black curve shows the model predicted $V$ given the body weight in all the subjects}
\label{fig: V_BW}
\end{figure}

\begin{figure}[h!]
\begin{center}
\includegraphics[width=15cm]{p.PNG}
\end{center}
\caption{Simulation of glucose concentration profiles of four virtual subjects in an IVGTT. The subjects have the same body weight, BMI, $p2$, $V$, with different $GEZI$ and $S_I$. Virtual subject A has normal $GEZI$ and $S_I$. Virtual subject B is a typical T2D patient with imparied $GEZI$ and $S_I$. For subject C, $GEZI$ is decreased but not $S_I$. $GEZI$ is unchanged in subject D, but $S_I$ is decreased to the same level as subject B}
\label{fig: simu}
\end{figure}

\end{document}


